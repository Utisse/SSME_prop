\documentclass[a4paper]{article}
\input{../preamble.tex}
\title{Appunti Tank}
\begin{document}
    \maketitle
    \tableofcontents
    \section{Volume del Tank}
    Il volume del tank deve essere maggiore di 
    quello del volume del propellente richiesto.

    Il tank è composto da 4 componenti:
    \begin{enumerate}
        \item $V_{pu}$: volume propellente (extra per emergenze).
        \item $V_{ull}$: volume _libero_ per espansione del propellente o contrazione strutturale (.01/0.03).
        \item $V_{bo}$: per criogenici, permette l'ebollizione a causa del rifonrnimento e dello scolo.
        \item $V_{trap}$: volume del propellente che rimane intrappolato nelle feed lines (tipicamente volume del feed system).  
    \end{enumerate}
    Dunque: 
    \[V_{tot} = V_{pu} + V_{ull} + V_{bo} + V_{trap}\]
    \section{Forma del Tank}
    Di solito sono: 
    \begin{itemize}
        \item Sferici
        \item Cilindrici
    \end{itemize}
    I primi massimizzano il volume rispetto all'area, ma i cilindrici hanno forma 
    migliore per razzi e migliorano la rigidità strutturale.
    
    La pressione ha il maggior effetto sui limiti strutturali del tank. Per il design
    preliminare consideriamo solo i carichi di pressione:

    La design burst pressure del tank è:
    \[p_b = f_s \text{MEOP}\]
    con: 
    \begin{itemize}
        \item $p_b$: design bust pressure (in Pa).
        \item $f_s$: fattore sicurezza (2.0 per veicoli pressurizzati)
        \item MEOP: Maximum Expected Operating Pressure del tank (Pa).
    \end{itemize}
    \question{Quale materiale? serve densità, sforzo}
\end{document}