\input{../preamble.tex}
\title{Appunti Tank}
\begin{document}
    \maketitle
    \tableofcontents
    \newpage
    \section{Volume del Tank}
    Il volume del tank deve essere maggiore di 
    quello del volume del propellente richiesto.

    Il tank è composto da 4 componenti:
    \begin{enumerate}
        \item $V_{pu}$: volume propellente (extra per emergenze).
        \item $V_{ull}$: volume \it{libero} per espansione del propellente o contrazione strutturale (.01/0.03).
        \item $V_{bo}$: per criogenici, permette l'ebollizione a causa del rifonrnimento e dello scolo.
        \item $V_{trap}$: volume del propellente che rimane intrappolato nelle feed lines (tipicamente volume del feed system).  
    \end{enumerate}
    Dunque: 
    \[V_{tot} = V_{pu} + V_{ull} + V_{bo} + V_{trap}\]
    \section{Forma del Tank}
    Di solito sono: 
    \begin{itemize}
        \item Sferici
        \item Cilindrici
    \end{itemize}
    I primi massimizzano il volume rispetto all'area, ma i cilindrici hanno forma 
    migliore per razzi e migliorano la rigidità strutturale.
    
    La pressione ha il maggior effetto sui limiti strutturali del tank. Per il design
    preliminare consideriamo solo i carichi di pressione:

    La design burst pressure del tank è:
    \[p_b = f_s \text{MEOP}\]
    con: 
    \begin{itemize}
        \item $p_b$: design bust pressure (in Pa).
        \item $f_s$: fattore sicurezza (2.0 per veicoli pressurizzati)
        \item $\text{MEPP}$: Maximum Expected Operating Pressure del tank (Pa).
    \end{itemize}
    \question{Quale materiale? serve densità ($\rho$), sforzo elastico ($F_{tu}$).}
    Indice di merito: \[\frac{F_{tu}}{\rho g_0}\]
    Altre cose da considerare nel materiale sono interazione 
    chimica e lavorazione (controllare in appendice B).
    \subsection{Sezione cilindrica}
    \begin{eqnarray*}
        t_s = \frac{p_b r_s}{2 F_{all}}\\
        m_s = A_s t_s \rho_{mat}
    \end{eqnarray*}
    con: 
    \begin{itemize}
        \item $r_s$ raggio della sfera
        \item $A_s$ superficie della sfera
        \item $V_s$ volume della sfera
        \item $t_s$ spessore
        \item $p_b$ DBP
        \item $F_{all}$ sforzo elastico permesso
        \item $m_s$ massa
        \item $\rho_{mat}$ densità del materiale
    \end{itemize}
    \subsection{Sezione sferica}
    \begin{eqnarray*}
        t_c = \frac{p_b r_c}{F_{all}}\\
        m_c = A_c t_c \rho_{mat}
    \end{eqnarray*}
    \subsubsection{Parti finali cilindriche}
    Non abbiamo ellissi perchè lo sforzo sul punto di contatto tra 
    ellissi e cilindro si formano sforzi alti. Abbiamo calotte sferiche.
    Non considerando i diversi loads dati da altri fattori che non siano la 
    pressione interna sottostimiamo di circa 2/2.5 volte.
    \section{Stima della massa con il metodo pV/W}
    Approccio puramente empirico. Considero $\phi_{tank}$, fattore della massa del tank:
    \begin{equation*}
        \phi_{tank} = \frac{p_b V_{tot}}{g_0 m_{tank}}
    \end{equation*}
    \emph{Per tank completamente metallici questo valore è 2500 metri!!}
    Posso quindi risolvere per la massa del tank.
    \subsection{Isolazione termica}
    Devo isolare liquidi criogenici o propellenti che potrebbero congelarsi in orbita. 
    Di solito questo è fatto tramite una lamina di metallo che copre una schiuma isolante (oppure materiale non metallico a nido d'ape).
    \subsection{Dispositivi di espulsione di cariburante}
    I serbatoi devono essere in grado di fornire e controllare il propellente in tutte le fasi della missione. 
    Parte di questa capacità deriva anche dalla rimozione di gas dal propellente che viene fornito al sistema propulsivo, e 
    espellere parte del propellente per eliminare i residui. Oltre a questo devono essere evitate forze trasmesse dal propellente alla navicella
    (slosh). Devo evitare che negli inlet finisca gas. 
    
    Posso avere: 
    \begin{itemize}
        \item sistemi di espulsione passivi
        \item sistemi di espulsione positivi
    \end{itemize}
    I primi meno costosi ed efficienti sfruttano la tensione superficiale del liquido, e sono dunque meno affidabili ma meno costosi e complicati, mentre i secondi usano barriere fisiche, valvole, pistoni, diaframmi etc.\ e sono quindi più efficaci a scapito di costo e complessità.
    \question{PAGINA 299 e precedenti}
    Posso inoltre usare gas per pressurizzare 
    \end{document}